\documentclass
[answers]
{exercise_sheet}

\newcommand\CourseDate{\today}
\newcommand\CourseInstitute{ETH Zurich}
\newcommand\CourseAbbreviation{OR}
\newcommand\CourseTitle{Operations Research in R}
\newcommand\CourseAuthor{Stefan Feuerriegel}

\newcommand{\CourseQuiz}{Visit webpage with course quiz.}
\newif\ifQuizSolution\QuizSolutionfalse
%%% Packages for LaTeX - programming

% Define commands that don't eat spaces.
\usepackage{xspace}

% IfThenElse
\usepackage{ifthen}

%%% Doc: ftp://tug.ctan.org/pub/tex-archive/macros/latex/contrib/oberdiek/ifpdf.sty
% command for testing for pdf-creation
\usepackage{ifpdf} %\ifpdf \else \fi
 
%%% Internal Commands: ----------------------------------------------
\makeatletter
%
\providecommand{\IfPackageLoaded}[2]{\@ifpackageloaded{#1}{#2}{}}
\providecommand{\IfPackageNotLoaded}[2]{\@ifpackageloaded{#1}{}{#2}}
\providecommand{\IfElsePackageLoaded}[3]{\@ifpackageloaded{#1}{#2}{#3}}
%
 
\providecommand{\IfDefined}[2]{%
\ifcsname #1\endcsname
   #2 %
\else
     % do nothing
\fi
}
%
% Check for 'draft' mode - commands.
\newcommand{\IfNotDraft}[1]{\ifdraft{}{#1}}
\newcommand{\IfNotDraftElse}[2]{\ifdraft{#2}{#1}}
\newcommand{\IfDraft}[1]{\ifdraft{#1}{}}

% Pakete speichern die spaeter geladen werden sollen
\newcommand{\LoadPackagesNow}{}
\newcommand{\LoadPackageLater}[1]{%
   \g@addto@macro{\LoadPackagesNow}{%
      \usepackage{#1}%
   }%
}
 
\makeatother
%%% ----------------------------------------------------------------
%% ------------------------------------------------------
%% These packages must be loaded prior to others
%% ------------------------------------------------------
 
% Calculation with LaTeX
\usepackage{calc}

%%% Info: http://www.mail-archive.com/ctan-ann@dante.de/msg01512.html
% For adding/using translations like "Theorem/Satz" inside babel, especially for beamer
\usepackage{translator}
 
% Images
\usepackage[%
  %final,
  %draft % do not include images (faster)
]{graphicx}

%% ------------------------------------------------------
%% Fonts
%% ------------------------------------------------------

\usepackage[T1]{fontenc} % T1 Schrift Encoding
\usepackage{textcomp}   % additional symbols (Text Companion font extension)


\usepackage{mathptmx}              %% --- Times with maths font
\usepackage[scaled=.90]{helvet}    %% --- Helvetica (Arial)
\usepackage{courier}               %% --- Courier

\renewcommand\familydefault{\sfdefault} 
\DeclareMathAlphabet      {\mathup}{OT1}{\familydefault}{m}{n}

%% ------------------------------------------------------
%% Maths
%% ------------------------------------------------------
 
% Extends and corrects amsmath
\usepackage[fixamsmath,disallowspaces]{mathtools}
 
% Warnings for commands that are incompatible with amsmath
\usepackage[
  all,
  warning
]{onlyamsmath}

% Using bold greek letters with \bf
\usepackage{bm}


%% ------------------------------------------------------
%% Units
%% ------------------------------------------------------
 
\usepackage{siunitx}
    \sisetup{
        detect-all,
        input-signs={+-<>\leq\geq\pm\mp\approx\sim},
        retain-explicit-plus,
    }

% Allows page breaks in mathemaical environments
\allowdisplaybreaks

%% ------------------------------------------------------
%% Customizations
%% ------------------------------------------------------

 
%  Exchange epsilon

\let\ORGvarepsilon=\varepsilon
\let\varepsilon=\epsilon
\let\epsilon=\ORGvarepsilon
%% ------------------------------------------------------
%% Symbols
%% ------------------------------------------------------

\usepackage{amssymb}
\usepackage[Symbolsmallscale]{upgreek} % upright symbols from euler package [Euler] or Adobe Symbols [Symbol]
\usepackage[upmu]{gensymb}             % Option upmu
 
%% General symbols
\usepackage{pifont}   %% 

%% http://tex.stackexchange.com/questions/42619/x-mark-to-match-checkmark
\newcommand{\cmark}{\ding{51}}%
\newcommand{\xmark}{\ding{55}}%

\usepackage{eurosym}
%% from: http://tex.stackexchange.com/questions/2441/how-to-add-a-forced-line-break-inside-a-table-cell
\newcommand{\mcell}[2][c]{%
  \begin{tabular}[c]{@{}#1@{}}#2\end{tabular}}
\newcommand{\mcellt}[2][c]{%
  \begin{tabular}[t]{@{}#1@{}}#2\end{tabular}}

\newcommand{\stackedcell}[2][c]{%
  \begin{tabular}[#1]{@{}c@{}}#2\end{tabular}}
%% ------------------------------------------------------
%% PDF-related packages
%% ------------------------------------------------------

\hypersetup{
   % Links
   raiselinks=true,       % calculate real height of the link
   breaklinks,             
   verbose,
   hyperindex=true,  
   linktocpage=true,      
   hyperfootnotes=false,  
   % Bookmarks
   bookmarksopenlevel=1,    
   bookmarksopen=false,   
   bookmarksnumbered=true, 
   bookmarkstype=toc,
   % Anchors
   plainpages=false, 
   pageanchor=true, 
   % PDF Informationen
   pdftitle={\CourseTitle},             % Titel
   pdfauthor={\CourseAuthor},            % Author
   pdfcreator={LaTeX, hyperref, beamer, pdfeTeX-0.\the\pdftexversion\pdftexrevision}, % Composer
   pdfstartview=Fit,  
   pdffitwindow=true,
   pdfpagemode=UseOutlines, % Show bookmarks in viewer
}
%% ------------------------------------------------------
%% Verbatim
%% ------------------------------------------------------

\usepackage{upquote} % Real quotes in verbatim environment
\usepackage{verbatim} % Reimplemntation of the original verbatim
 
%% ------------------------------------------------------
%% Listings package 
%% ------------------------------------------------------

\usepackage{listings}
  \lstset{%
    basicstyle=\scriptsize\ttfamily,
    tabsize=3,
    showtabs=false,
    showspaces=false,
    showstringspaces=false,
    tab=\rightarrowfill,
    keywordstyle=\color{ListingsKeywordColor},
    commentstyle=\color{ListingsCommentColor},
    stringstyle=\color{ListingsStringColor},
    frame=none,
    rulesepcolor=\color{ListingsRuleSepColor},
    numbers=none, 
    numberstyle=\tiny, 
    numbersep=5pt,
    captionpos=top,
    firstnumber=1,
    stepnumber=5,
    numberfirstline=false,
    breaklines=true,
    morekeywords={}
}

  \lstloadlanguages{R}

%% ------------------------------------------------------
%% Algorithm environments
%% ------------------------------------------------------

\usepackage{algorithm}

\usepackage{algorithmic}        % for pseudocode
  \renewcommand{\algorithmicrequire}{\textbf{Input:}}
  \renewcommand{\algorithmicensure}{\textbf{Output:}}
  
  \makeatletter
\newenvironment{inlinealgorithm}[1]{%
  \def\@fs@cfont{\bfseries}%
  \let\@fs@capt\relax%
  \par\noindent%
  %\medskip%
  \rule{\linewidth}{.8pt}%
  \vspace{-10pt}%
  \noindent\captionof{algorithm}{#1}%
  \vspace{-0.7\baselineskip}%
  \noindent\rule{\linewidth}{.4pt}%
  \vspace{-1.3\baselineskip}%
  %% My hack, as algorithmic uses the overall number of lines in the whole document for labels, instead of using the description from within a single algorithm
  \renewcommand*\theALC@unique{\theALC@line}
}{%
  \vspace{-.75\baselineskip}%
  \rule{\linewidth}{.4pt}%
  \medskip%
}
\makeatother
%% ------------------------------------------------------
%% Diagrams
%% ------------------------------------------------------

\usepackage{smartdiagram}

%% ------------------------------------------------------
%% Animations
%% ------------------------------------------------------

\usepackage{animate}
%% ------------------------------------------------------
%% Quotes
%% ------------------------------------------------------

\usepackage[%
   babel=true,            % the style of all quotation marks will be adapted
                     % to the document language as chosen by 'babel'
   german=guillemets,    % Styles of quotes in each language
   english=british,
   french=guillemets
%   style=guillemets
]{csquotes}


\mode<presentation>
{
  \usetheme{Teaching}
  \beamertemplatenavigationsymbolsempty 

\setbeamertemplate{sections/subsections in toc}[square]
\setbeamertemplate{enumerate items}[square] % Places the numbers on little squares

  \setbeamercovered{transparent}
  
 \renewcommand\boldmath{\mathversion{bold}} %% Fix conflict with package bm inside mathptmx
}
   
\AtBeginSubsection[]
{
  \begin{frame}<presentation>{Outline}
    \tableofcontents[sectionstyle=show/hide,subsectionstyle=show/shaded/hide]
  \end{frame}
}

\AtBeginSection[]
{
  \begin{frame}<beamer>{Outline}
    \tableofcontents[sectionstyle=show/shaded,subsectionstyle=hide]
  \end{frame}
}

\mode<handout>
{
  \usepackage{pgfpages}
  \pgfpagesuselayout{4 on 1}[a4paper,border shrink=5mm]
}

\title[\CourseTopicShort]{\CourseTopic}
\author{\CourseAuthor}
\date[]{\CourseDate}
\institute{\CourseInstitute}

\graphicspath{
      {img/}
      {../../Graphics/PDF_PNG/}
  }


\hyphenation{}  

\listfiles



<<setup, include=FALSE>>=
# change default language to display error messages in English
Sys.setenv(LANGUAGE = "en")

# smaller font size for chunks
library(knitr)
opts_chunk$set(size = "footnotesize", fig.width = 3, fig.height = 3)
# , highlight=FALSE

listing_source <- function(x, options) {
    paste("\\parbox{\\textwidth}{",
      x, "}\n\n", sep = "")
  }
listing_output <- function(x, options) {
    paste("\\parbox{\\textwidth}{",
      x, "}\n\n", sep = "")
  }
  
# knit_hooks$set(source=listing_source, output=listing_output)

# , inline = function(x, options) x, document = function(x, options) x, chunk = function(x, options) x
# \\parbox{\\textwidth}{",
@

\lohead{\CourseTitle: Homework 3}

%% necessary for knitr
%\documentclass{article}
%\renewenvironment{alltt}{}{}

\begin{document}

\section*{Elementary Concepts in Linear Algebra}

\subsection*{Number Representations}

\begin{Question}{1}
Convert $1\,110\,000\,101$ from base $2$ into base $10$ by hand. 
\end{Question}

\makeatletter\if@answers\begin{Answer}{3}
\begin{align*}
& 1 \cdot 2^8 + 1 \cdot 2^7 + 1 \cdot 2^6 + 0 \cdot 2^5 + 0 \cdot 2^4 + 0 \cdot 2^3 + 1 \cdot 2^2 + 0 \cdot 2^1 + 1 \\
= & 1 \cdot 256 + 1 \cdot 128 + 1 \cdot 64 + 0 \cdot 32 + 0 \cdot 16 + 0 \cdot 8 + 1 \cdot 4 + 0 \cdot 2 + 1 \\
= & 901 \text{ in base 10}
\end{align*}
\end{Answer}\fi\makeatother

\begin{Question}{1}
Convert $852$ from base $10$ into base $2$ by hand. 
\end{Question}

\makeatletter\if@answers\begin{Answer}{8}
\begin{tabular}{rrr}
\textbf{Start} & \textbf{Integer division by 2} & \textbf{Remainder} \\ \midrule
852 & 426 & 0 \\
426 & 213 & 0 \\
213 & 106 & 1 \\
106 & 53 & 0 \\
53 & 26 & 1 \\
26 & 13 & 0 \\
13 & 6 & 1\\
6 & 3 & 0\\
3 & 1 & 1 \\
1 & 0 & 1 
\end{tabular}

Result: $1\,101\,010\,100$ in base 2
\end{Answer}\fi\makeatother

\begin{Question}{3}
Repeat the above base 2 conversions by utilizing R.  
\end{Question}

\makeatletter\if@answers\begin{Answer}{3}
<<>>=
library(sfsmisc)
strtoi(1110000101, base=2) 
digitsBase(852, base=2)
@
\end{Answer}\fi\makeatother

\begin{Question}{1}
Convert $756\,001$ from base $8$ into base $10$ by hand. 
\end{Question}

\makeatletter\if@answers\begin{Answer}{3}
\begin{align*}
& 7 \cdot 8^5 + 5 \cdot 8^4 + 6 \cdot 8^3 + 0 \cdot 8^2 + 0 \cdot 8^1 + 1 \\
& 7 \cdot 32768 + 5 \cdot 4096 + 6 \cdot 512 + 0 \cdot 64 + 0 \cdot 8 + 1 \\
= & 252\,929 \text{ in base 10}
\end{align*}
\end{Answer}\fi\makeatother

\begin{Question}{1}
Convert $9530$ from base $10$ into base $8$ by hand. 
\end{Question}

\makeatletter\if@answers\begin{Answer}{8}
\begin{tabular}{rrr}
\textbf{Start} & \textbf{Integer division by 8} & \textbf{Remainder} \\ \midrule
9530 & 1191 & 2 \\
1191 & 148 & 7 \\
148 & 18 & 4 \\
18 & 2 & 2 \\
2 & 0 & 2 \\
\end{tabular}

Result: $22\,472$ in base 8
\end{Answer}\fi\makeatother

\begin{Question}{3}
Repeat the above base 2 conversions by utilizing R.  
\end{Question}

\makeatletter\if@answers\begin{Answer}{3}
<<>>=
library(sfsmisc)
strtoi(756001, base=8) 
digitsBase(9530, base=8)
@
\end{Answer}\fi\makeatother

\begin{Question}{10}
Write a function \verb|dec_to_bin| that takes a decimal number as input and outputs its binary representation in a vector of $1$s and $0$s. Instead of using the build-in conversion from R, implement this with help of an appropriate loop. Then test its correctness by calculating the binary representation of $852$.
\end{Question}

\makeatletter\if@answers\begin{Answer}{15}
<<>>=
dec_to_bin <- function(decimal) {
  binary <- c()
  while (decimal > 0) {
    if (decimal %% 2 == 0) {
      binary <- c(0, binary)
    } else {
      binary <- c(1, binary)
    }
    decimal <- floor(decimal / 2)
  }
	
  return(binary)
}

dec_to_bin(852) # should return 1,101,010,100
@
\end{Answer}\fi\makeatother

\subsection*{Polynomials}

\begin{Question}{4}
Write a function \verb|my_power| in R that calculates the exponentiation of two numbers $b^n$ using a for-loop. Assume that $n \in \mathbb{N}$. Finally, use this function to calculate $7^3$ and $5^0$.
\end{Question}

\makeatletter\if@answers\begin{Answer}{15}
<<>>=
my_power <- function(base, exponent) {
  result <- 1
  if (exponent != 0) {
    for(i in 1:exponent) {
      result <- result * base
    }
  }
  return(result)
}

my_power(5, 0)  # should return 1
my_power(17, 3) # should return 343
@
\end{Answer}\fi\makeatother


\begin{Question}{4}
Create a function \verb|horner| that evaluates a polynomial at given point $x_0 = 3$ by utilizing the Horner scheme for a given vector of polynomial coefficients. For example, the polynomial $2x^4 - 4x^3 - 5x^2 + 7x + 11$ corresponds to a vector representation \verb|c(2, -4, -5, 7, 11)| in R. 
% Function change, no longer relevant: The function should return a string representing the correct Horner form of the polynomial.
\end{Question}

\makeatletter\if@answers\begin{Answer}{10}
<<>>=
horner <- function(coeffs, x0) {
	value <- 0
	
	for (i in 1:length(coeffs)) {
	  value <- value * x0 + coeffs[i]
	}
	
	return(value)
}
	
horner(c(2, -4, -5, 7, 11), 3)

# double-check with built-in R
2*3^4 - 4*3^3 - 5*3^2 + 7*3 + 11
@
\end{Answer}\fi\makeatother

\subsection*{Linear Algebra}

\begin{Question}{2}
Calculate the dot product for the following two vectors by hand and with the help of R.
\begin{equation*}
\vecval{x} = 
\begin{bmatrix} 
5 \\ -1 \\ 2 
\end{bmatrix}
\quad
\text{and}
\quad
\vecval{y} = 
\begin{bmatrix} 
8 \\ 7 \\ -3 
\end{bmatrix}
.
\end{equation*}
\end{Question}

\makeatletter\if@answers\begin{Answer}{5}
\begin{equation*}
\begin{bmatrix} 
5 \\ -1 \\ 2 
\end{bmatrix}
\cdot 
\begin{bmatrix} 
8 \\ 7 \\ -3 
\end{bmatrix} 
= 5 \cdot 8 + -1 \cdot 7 + 2 \cdot -3 
= 40 - 7 - 6 
= 27 
\end{equation*}
\end{Answer}

<<>>=
drop(c(5, -1, 2) %*% c(8, 7, -3))
@
\fi\makeatother

\begin{Question}{2}
Write a function \verb|dot_product| in R to calculate the dot product of the vectors from the previous exercise. Use a for loop to calculate the result.
\end{Question}

\makeatletter\if@answers\begin{Answer}{12}

<<>>=
dot_product <- function(vector1, vector2) {
  result <- 0
  for (i in 1:length(vector1)) {
    result <- result + vector1[i]*vector2[i]
  }
  return(result)
}

x <- c(5, -1, 2)
y <- c(8, 7, -3)

dot_product(x,y)

@
\end{Answer}\fi\makeatother

\begin{Question}{2}
Write a function \verb|mult_matrix_vector| in R to multiply a matrix by a vector. Use a for loop to calculate the result. Apply your function to $A$ and $\vecval{x}$ defined by
\begin{equation*}
A = 
\begin{bmatrix}
4 & 3 \\ 
-1 & 7 \\ 
0 & 12 
\end{bmatrix}
\quad
\text{and}
\quad
\vecval{x} = 
\begin{bmatrix} 
3 \\ -2 
\end{bmatrix}
\end{equation*}
and then compare to the built-in functionality within R.
\end{Question}

\makeatletter\if@answers\begin{Answer}{12}
<<>>=  
mult_matrix_vector <- function(matrix, vector) {
  result <- c()
  for (row_i in 1:nrow(matrix)) {
    row <- matrix[row_i, ]
    row_result <- dot_product(row, vector)
    result <- c(result, row_result)
  }
  return(result)
}

A <- matrix(c(4, -1, 0, 3, 7, 12), nrow=3, ncol=2)
x <- c(3, -2)

mult_matrix_vector(A, x)

mult_matrix_vector(A, x) - A %*% x # post check
@
The result should be $A\vecval{x} = \begin{bmatrix} 6 \\ -17 \\ -24 \end{bmatrix}$.
\end{Answer}\fi\makeatother

\begin{Question}{2}
Write a function \verb|mult_matrix_matrix| in R to multiply two matrices. Use a for loop to calculate the result. Apply your function to $A$ and $B$ with
\begin{equation*}
A = 
\begin{bmatrix} 
4 & 3 \\ 
-1 & 7 \\ 
0 & 12 
\end{bmatrix} 
\quad
\text{and}
\quad
B = \begin{bmatrix} 
3 & 2 & 5 & 1\\ 
-2 & 0 & 4 & 1
\end{bmatrix}
.
\end{equation*}

Hint: The result should be
$AB = \begin{bmatrix}
6 & 8 & 32 & 7 \\
-17 & -2 & 23 & 6 \\
-24 & 0 & 48 & 12
\end{bmatrix}$.

\end{Question}

\makeatletter\if@answers\begin{Answer}{12}
<<>>=  
mult_matrix_matrix <- function(matrix1, matrix2) {
  result_rows <- nrow(matrix1)
  result_cols <- ncol(matrix2)
  result <- c()
  for (col_i in 1:ncol(matrix2)) {
    col <- matrix2[, col_i]
    col_result <- mult_matrix_vector(matrix1, col)
    result <- c(result, col_result)
  }
  return(matrix(result, nrow=result_rows, ncol=result_cols))
}

A <- matrix(c(4, -1, 0, 3, 7, 12), nrow=3, ncol=2)
B <- matrix(c(3, -2, 2, 0, 5, 4, 1, 1), nrow=2, ncol=4)

mult_matrix_matrix(A, B)
@
\end{Answer}\fi\makeatother

\begin{Question}{3}
Consider the vector 
\begin{equation*}
\vecval{x} = \begin{bmatrix} 
-3 \\ -4 \\ -1 \\ 0 \\ 2
\end{bmatrix}.  
\end{equation*}
What are its $L^1$-, $L^2$ and $4$-norm?
\end{Question}

\makeatletter\if@answers\begin{Answer}{4}
<<>>=
x <- c(-3, -4, -1, 0, 2)

sum(abs(x)) # L1-norm
sqrt(sum(x^2)) # L2-norm

(sum(abs(x)^4))^(1/4) # 4-norm
@
\end{Answer}\fi\makeatother

\begin{Question}{2}
Calculate the inverse (or pseudoinverse, depending on what is necessary) for the following matrix. Double-check subsequently.
\begin{equation*}
A = \begin{bmatrix}
3 & 7 \\
-1 & 2
\end{bmatrix}
\quad
\text{and}
\quad
B = \begin{bmatrix}
2 & 8 \\
3 & 12
\end{bmatrix}
\end{equation*}
\end{Question}

\makeatletter\if@answers\begin{Answer}{8}
<<>>=
library(MASS)

A <- matrix(c(3,-1, 7,2), ncol=2)
A
det(A)  # != 0, thus A is invertible
solve(A)  # inverse
A %*% solve(A) - diag(2) # post check

B <- matrix(c(2,3, 8,12), ncol=2)
B
det(B)  # == 0, thus B is not invertible
ginv(B)  # pseudo-inverse
B*ginv(B) - diag(2)  # post-check
@
\end{Answer}\fi\makeatother

\begin{Question}{1}
Find the determinant, eigenvalues and eigenvectors to the matrix 
\begin{equation*}
A = 
\begin{bmatrix}
3 & 3 \\
1 & 1
\end{bmatrix}
\end{equation*}
by hand.
\end{Question}

\makeatletter\if@answers\begin{Answer}{2}
To find the eigenvalues, we solve the characteristic equation
\begin{align*}
& \det{A - \lambda I} = 0 \\
\Leftrightarrow & \det \begin{bmatrix}
3-\lambda & 3 \\
1 & 1-\lambda
\end{bmatrix} = 0 \\
\Leftrightarrow & 3 - 4\lambda + \lambda^2 - 3 = 0 \\
\Rightarrow & \lambda_1 = 4, \lambda_2 = 0
\end{align*}

To find the corresponding eigenvectors, we solve the equation $(A - \lambda I) \vecval{x} = 0$. Given $\lambda_1 = 4$, we obtain
\begin{align*}
&
\begin{bmatrix}
3-4 & 3 \\
1 & 1-4
\end{bmatrix}
\cdot 
\begin{bmatrix}
x_1 \\
x_2
\end{bmatrix}
=
\begin{bmatrix}
-1 & 3 \\
1 & -3
\end{bmatrix}
\cdot 
\begin{bmatrix}
x_1 \\
x_2
\end{bmatrix}
=
\begin{bmatrix}
0 \\
0
\end{bmatrix}
\\
\Rightarrow & -1 x_1 + 3 x_2 = 0 \\
\Leftrightarrow & x_1 = 3 x_2 \\
\Rightarrow & \vecval{v}_1 = \begin{bmatrix}
1 \\
3
\end{bmatrix}
.
\end{align*}

Given $\lambda_2 = 0$, we obtain
\begin{align*}
&
\begin{bmatrix}
3-0 & 3 \\
1 & 1-0
\end{bmatrix}
\cdot 
\begin{bmatrix}
x_1 \\
x_2
\end{bmatrix}
=
\begin{bmatrix}
3 & 3 \\
1 & 1
\end{bmatrix}
\cdot 
\begin{bmatrix}
x_1 \\
x_2
\end{bmatrix}
=
\begin{bmatrix}
0 \\
0
\end{bmatrix}
\\
\Rightarrow & 3 x_1 + 3 x_2 = 0 \\
\Leftrightarrow & x_1 = -1 x_2 \\
\Rightarrow & \vecval{v}_2 = \begin{bmatrix}
1 \\
-1
\end{bmatrix} 
.
\end{align*}
\end{Answer}\fi\makeatother

\begin{Question}{1}
Calculate the determinates, eigenvalues and eigenvectors of the two matrices
\begin{equation*}
A_1 =
\begin{bmatrix}
3 & 3 \\
1 & 1
\end{bmatrix}
\quad
\text{and}
\quad
A_2 = 
\begin{bmatrix}
1 & 2 \\
0 & 1
\end{bmatrix}
\end{equation*}
with the help of R.
\end{Question}

\makeatletter\if@answers\begin{Answer}{2}
<<>>=
A1 <- matrix(c(3, 1, 3, 1), nrow=2)
A1
det(A1)
eigen(A1)

A2 <- matrix(c(1, 0, 2, 1), nrow=2)
A2
det(A2)
eigen(A2)
@
\end{Answer}\fi\makeatother

\begin{Question}{1}
Are the following two matrices positive or negative definite?
\begin{equation*}
A_1 =
\begin{bmatrix}
3 & -1 \\
-1 & 3
\end{bmatrix}
\quad
\text{and}
\quad
A_2 = 
\begin{bmatrix}
1 & 0 \\
0 & 2
\end{bmatrix}
\end{equation*}
\end{Question}

\makeatletter\if@answers\begin{Answer}{2}
<<>>=
library(matrixcalc)

A1 <- matrix(c(3, -1, -1, 3), nrow=2)
A1
is.negative.definite(A1)
is.positive.definite(A1)

A2 <- matrix(c(1, 0, 0, 2), nrow=2)
A2
is.negative.definite(A2)
is.positive.definite(A2)
@
\end{Answer}\fi\makeatother

\end{document}